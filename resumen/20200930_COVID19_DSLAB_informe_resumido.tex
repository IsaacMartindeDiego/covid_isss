% Options for packages loaded elsewhere
\PassOptionsToPackage{unicode}{hyperref}
\PassOptionsToPackage{hyphens}{url}
%
\documentclass[
  11pt,
]{article}
\usepackage{lmodern}
\usepackage{amssymb,amsmath}
\usepackage{ifxetex,ifluatex}
\ifnum 0\ifxetex 1\fi\ifluatex 1\fi=0 % if pdftex
  \usepackage[T1]{fontenc}
  \usepackage[utf8]{inputenc}
  \usepackage{textcomp} % provide euro and other symbols
\else % if luatex or xetex
  \usepackage{unicode-math}
  \defaultfontfeatures{Scale=MatchLowercase}
  \defaultfontfeatures[\rmfamily]{Ligatures=TeX,Scale=1}
  \setmainfont[]{Roboto Light}
\fi
% Use upquote if available, for straight quotes in verbatim environments
\IfFileExists{upquote.sty}{\usepackage{upquote}}{}
\IfFileExists{microtype.sty}{% use microtype if available
  \usepackage[]{microtype}
  \UseMicrotypeSet[protrusion]{basicmath} % disable protrusion for tt fonts
}{}
\makeatletter
\@ifundefined{KOMAClassName}{% if non-KOMA class
  \IfFileExists{parskip.sty}{%
    \usepackage{parskip}
  }{% else
    \setlength{\parindent}{0pt}
    \setlength{\parskip}{6pt plus 2pt minus 1pt}}
}{% if KOMA class
  \KOMAoptions{parskip=half}}
\makeatother
\usepackage{xcolor}
\IfFileExists{xurl.sty}{\usepackage{xurl}}{} % add URL line breaks if available
\IfFileExists{bookmark.sty}{\usepackage{bookmark}}{\usepackage{hyperref}}
\hypersetup{
  pdftitle={Informe del 01 de octubre de 2020},
  hidelinks,
  pdfcreator={LaTeX via pandoc}}
\urlstyle{same} % disable monospaced font for URLs
\usepackage[margin=1in]{geometry}
\usepackage{graphicx,grffile}
\makeatletter
\def\maxwidth{\ifdim\Gin@nat@width>\linewidth\linewidth\else\Gin@nat@width\fi}
\def\maxheight{\ifdim\Gin@nat@height>\textheight\textheight\else\Gin@nat@height\fi}
\makeatother
% Scale images if necessary, so that they will not overflow the page
% margins by default, and it is still possible to overwrite the defaults
% using explicit options in \includegraphics[width, height, ...]{}
\setkeys{Gin}{width=\maxwidth,height=\maxheight,keepaspectratio}
% Set default figure placement to htbp
\makeatletter
\def\fps@figure{htbp}
\makeatother
\setlength{\emergencystretch}{3em} % prevent overfull lines
\providecommand{\tightlist}{%
  \setlength{\itemsep}{0pt}\setlength{\parskip}{0pt}}
\setcounter{secnumdepth}{-\maxdimen} % remove section numbering
\usepackage{titling}
\usepackage{fancyhdr}
\usepackage{lastpage}



\topmargin=-0.45in
\evensidemargin=-1in
\oddsidemargin=0in
\textwidth=6.5in
\textheight=9.0in
\headsep=0.25in

\linespread{1}
 \pagestyle{fancy}
 \fancyhead[LO,LE]{\thetitle}
 \fancyhead[RO,RE]{\includegraphics[width=0.5cm]{./DSLab_logo_2.png}}
 \fancyfoot[LO,LE]{DSLab, methaodos.org, Academia Joven de España \thedate}
 \cfoot{}
 \fancyfoot[RE,RO]{Página\ \thepage\ de\ \protect\pageref{LastPage}}

\renewcommand\headrulewidth{0.4pt}
\renewcommand\footrulewidth{0.4pt}


\pretitle{%
  \begin{center}
  \LARGE
  \vspace{-2.5cm}
  \includegraphics[width=3.2cm]{./methaodos.png}
  \hspace{3cm}
  \includegraphics[width=3.5cm]{./dslab.png}
  \hspace{3cm}
  \includegraphics[width=3cm]{./aje.png}\\ 
\vspace{1cm}
\textbf{Suficiencia Sanitaria y COVID-19}\\
}
\posttitle{\end{center}}
\usepackage[font=small,skip=8pt]{caption}
\usepackage{float}
\floatplacement{figure}{H}

\title{Informe del 01 de octubre de 2020}
\author{}
\date{\vspace{-2.5em}}

\begin{document}
\maketitle

\renewcommand{\figurename}{Figura}
\renewcommand{\tablename}{Tabla}

\vspace{-0.5cm}

\hypertarget{resumen-ejecutivo}{%
\subsection{Resumen ejecutivo}\label{resumen-ejecutivo}}

\hypertarget{indicador-de-suficiencia-sanitaria}{%
\subsection{Indicador de Suficiencia
Sanitaria}\label{indicador-de-suficiencia-sanitaria}}

El \textbf{Indicador de Suficiencia Sanitaria (ISS)} mide la capacidad
que tiene un sistema de salud para dar respuesta a las necesidades
derivadas de un creciente número de contagios en un escenario de
pandemia como es la del COVID-19. Valores altos del ISS indicarán que el
sistema sanitario tiene la capacidad de absorber el flujo de
diagnósticos confirmados de coronavirus al objeto de que puedan ser
tratados según los protocolos establecidos. Valores bajos del ISS se
corresponden con posibles insuficiencias del sistema sanitario que
pueden afectar a la gestión del flujo de casos diagnosticados. Se define
un día positivo como aquel en el que el ISS diario toma un valor por
encima de 1.

\vspace{0.2cm}

\begin{figure}
\centering
\includegraphics{20200930_COVID19_DSLAB_informe_resumido_files/figure-latex/fig_iss-1.pdf}
\caption{\label{fig:fig_iss} ISS acumulado y semanal.}
\end{figure}

Los valores de ISS acumulado en España en las tres últimas semanas son:

\begin{itemize}
\tightlist
\item
  Última semana: 82.54\%
\item
  Penúltima semana: 80.39\%
\item
  Antepenúltima semana: 77.88\%
\end{itemize}

Los valores de ISS semanal en España en las tres últimas semanas ha
sido:

\begin{itemize}
\tightlist
\item
  Última semana: 1.02
\item
  Penúltima semana: 0.97
\item
  Antepenúltima semana: 0.9
\end{itemize}

\hypertarget{segunda-ola}{%
\subsection{Segunda Ola}\label{segunda-ola}}

\begin{figure}
\centering
\includegraphics{20200930_COVID19_DSLAB_informe_resumido_files/figure-latex/fig_iss2-1.pdf}
\caption{\label{fig:fig_iss2} ISS acumulado segunda ola.}
\end{figure}

\clearpage

\hypertarget{comparativa-por-comunidades-autuxf3nomas}{%
\subsection{Comparativa por Comunidades
Autónomas}\label{comparativa-por-comunidades-autuxf3nomas}}

La siguiente tabla muestra los valores de ISS acumulado e ISS semanal
para las Comunidades Autónomas españolas. \vspace{0.2cm}

\begin{table}[!h]

\caption{\label{tab:tabla}ISS acumulado e ISS semanal por CCAA}
\centering
\fontsize{9}{11}\selectfont
\begin{tabular}[t]{l|r|r}
\hline
Comunidad Autónoma & ISS acumulado (\%) & ISS semanal\\
\hline
Andalucia & 80.66 & 1.12\\
\hline
Aragón & 84.88 & 0.02\\
\hline
Asturias & 71.24 & 0.58\\
\hline
Baleares & 100.00 & 1.48\\
\hline
C. Valenciana & 83.83 & 1.38\\
\hline
Canarias & 90.83 & 2.32\\
\hline
Cantabria & 74.53 & 1.01\\
\hline
Castilla La Mancha & 67.14 & 0.86\\
\hline
Castilla y León & 75.24 & 0.76\\
\hline
Cataluña & 84.11 & 1.51\\
\hline
Ceuta & 82.29 & 1.53\\
\hline
Extremadura & 65.95 & 0.73\\
\hline
Galicia & 92.82 & 1.70\\
\hline
La Rioja & 86.78 & 1.38\\
\hline
Madrid & 76.24 & 0.82\\
\hline
Melilla & 68.04 & 1.11\\
\hline
Murcia & 66.94 & 0.94\\
\hline
Navarra & 66.59 & 0.67\\
\hline
País Vasco & 88.87 & 1.46\\
\hline
España & 80.00 & 0.99\\
\hline
\end{tabular}
\end{table}

Estos gráficos representan la evolución del ISS acumulado y diario en
las Comunidades Autónomas españolas.

\vspace{0.2cm}

\begin{figure}
\centering
\includegraphics{20200930_COVID19_DSLAB_informe_resumido_files/figure-latex/fig17a_res-1.pdf}
\caption{\label{fig:fig17a_res} ISS acumulado en las Comunidades
Autónomas}
\end{figure}

\begin{figure}
\centering
\includegraphics{20200930_COVID19_DSLAB_informe_resumido_files/figure-latex/figura_ISSSD_ccaa-1.pdf}
\caption{\label{fig:figura_ISSSD_ccaa} ISS semanal por Comunidades
Autónomas}
\end{figure}

\vspace{0.2cm}

\begin{figure}
\centering
\includegraphics{20200930_COVID19_DSLAB_informe_resumido_files/figure-latex/fig17a_res2-1.pdf}
\caption{\label{fig:fig17a_res} ISS acumulado en las Comunidades
Autónomas. Segunda ola.}
\end{figure}

\clearpage

\hypertarget{activos}{%
\subsection{Activos}\label{activos}}

Estos gráficos muestran los diagnosticados que permanecen activos cada
semana en España y en las CCAA. Se considera que la pandemia ha superado
su pico máximo cuando el número de activos semanales comienza a
descender de manera continuada.

\vspace{0.2cm}

\begin{figure}
\centering
\includegraphics{20200930_COVID19_DSLAB_informe_resumido_files/figure-latex/fig1-1.pdf}
\caption{\label{fig:fig17a_res} Activos en España desde el inicio de la
pandemia}
\end{figure}

\begin{figure}
\centering
\includegraphics{20200930_COVID19_DSLAB_informe_resumido_files/figure-latex/figura_activos_ccaa-1.pdf}
\caption{\label{fig:figura_activos_ccaa} Activos por Comunidades
Autónomas}
\end{figure}

\begin{center}
\includegraphics{cc.png}
\end{center}

\vspace{0.2cm}

\begin{center}
Esta obra está bajo una licencia de Creative Commons 
\end{center}

\begin{center}
Reconocimiento-CompartirIgual 4.0 Internacional.
\end{center}

\end{document}
