% Options for packages loaded elsewhere
\PassOptionsToPackage{unicode}{hyperref}
\PassOptionsToPackage{hyphens}{url}
%
\documentclass[
  11pt,
]{article}
\usepackage{lmodern}
\usepackage{amssymb,amsmath}
\usepackage{ifxetex,ifluatex}
\ifnum 0\ifxetex 1\fi\ifluatex 1\fi=0 % if pdftex
  \usepackage[T1]{fontenc}
  \usepackage[utf8]{inputenc}
  \usepackage{textcomp} % provide euro and other symbols
\else % if luatex or xetex
  \usepackage{unicode-math}
  \defaultfontfeatures{Scale=MatchLowercase}
  \defaultfontfeatures[\rmfamily]{Ligatures=TeX,Scale=1}
  \setmainfont[]{Roboto Light}
\fi
% Use upquote if available, for straight quotes in verbatim environments
\IfFileExists{upquote.sty}{\usepackage{upquote}}{}
\IfFileExists{microtype.sty}{% use microtype if available
  \usepackage[]{microtype}
  \UseMicrotypeSet[protrusion]{basicmath} % disable protrusion for tt fonts
}{}
\makeatletter
\@ifundefined{KOMAClassName}{% if non-KOMA class
  \IfFileExists{parskip.sty}{%
    \usepackage{parskip}
  }{% else
    \setlength{\parindent}{0pt}
    \setlength{\parskip}{6pt plus 2pt minus 1pt}}
}{% if KOMA class
  \KOMAoptions{parskip=half}}
\makeatother
\usepackage{xcolor}
\IfFileExists{xurl.sty}{\usepackage{xurl}}{} % add URL line breaks if available
\IfFileExists{bookmark.sty}{\usepackage{bookmark}}{\usepackage{hyperref}}
\hypersetup{
  pdftitle={Informe del 01 de noviembre de 2020},
  hidelinks,
  pdfcreator={LaTeX via pandoc}}
\urlstyle{same} % disable monospaced font for URLs
\usepackage[margin=1in]{geometry}
\usepackage{graphicx,grffile}
\makeatletter
\def\maxwidth{\ifdim\Gin@nat@width>\linewidth\linewidth\else\Gin@nat@width\fi}
\def\maxheight{\ifdim\Gin@nat@height>\textheight\textheight\else\Gin@nat@height\fi}
\makeatother
% Scale images if necessary, so that they will not overflow the page
% margins by default, and it is still possible to overwrite the defaults
% using explicit options in \includegraphics[width, height, ...]{}
\setkeys{Gin}{width=\maxwidth,height=\maxheight,keepaspectratio}
% Set default figure placement to htbp
\makeatletter
\def\fps@figure{htbp}
\makeatother
\setlength{\emergencystretch}{3em} % prevent overfull lines
\providecommand{\tightlist}{%
  \setlength{\itemsep}{0pt}\setlength{\parskip}{0pt}}
\setcounter{secnumdepth}{-\maxdimen} % remove section numbering
\usepackage{titling}
\usepackage{fancyhdr}
\usepackage{lastpage}



\topmargin=-0.45in
\evensidemargin=-1in
\oddsidemargin=0in
\textwidth=6.5in
\textheight=9.0in
\headsep=0.25in

\linespread{1}
 \pagestyle{fancy}
 \fancyhead[LO,LE]{\thetitle}
 \fancyhead[RO,RE]{\includegraphics[width=0.5cm]{./DSLab_logo_2.png}}
 \fancyfoot[LO,LE]{DSLab, methaodos.org, Academia Joven de España \thedate}
 \cfoot{}
 \fancyfoot[RE,RO]{Página\ \thepage\ de\ \protect\pageref*{LastPage}}

\renewcommand\headrulewidth{0.4pt}
\renewcommand\footrulewidth{0.4pt}


\pretitle{%
  \begin{center}
  \LARGE
  \vspace{-2.5cm}
  \includegraphics[width=3.2cm]{./methaodos.png}
  \hspace{3cm}
  \includegraphics[width=3.5cm]{./dslab.png}
  \hspace{3cm}
  \includegraphics[width=3cm]{./aje.png}\\ 
\vspace{1cm}
\textbf{Suficiencia Sanitaria y COVID-19}\\
}
\posttitle{\end{center}}
\usepackage[font=small,skip=8pt]{caption}
\usepackage{float}
\floatplacement{figure}{H}
\renewcommand{\contentsname}{Índice}
\usepackage{booktabs}
\usepackage{longtable}
\usepackage{array}
\usepackage{multirow}
\usepackage{wrapfig}
\usepackage{float}
\usepackage{colortbl}
\usepackage{pdflscape}
\usepackage{tabu}
\usepackage{threeparttable}
\usepackage{threeparttablex}
\usepackage[normalem]{ulem}
\usepackage{makecell}
\usepackage{xcolor}

\title{Informe del 01 de noviembre de 2020}
\author{}
\date{\vspace{-2.5em}}

\begin{document}
\maketitle

\renewcommand{\figurename}{Figura}
\renewcommand{\tablename}{Tabla}

\tableofcontents

\vspace{.5cm}

\begin{center}\rule{0.5\linewidth}{0.5pt}\end{center}

\vspace{.5cm}

Para analizar el comportamiento de la segunda ola de la pandemia
provocada por el virus SARS-CoV-2 y la enfermedad COVID-19 en España, se
han diseñado dos \textbf{Indicadores de Suficiencia Sanitaria (ISS):}
uno acumulado \textbf{(ISSa)} y otro semanal \textbf{(ISSs)}. Además, se
lleva a cabo una estimación del número de \textbf{casos activos} desde
que se inició la pandemia.

\hypertarget{indicador-de-suficiencia-sanitaria-acumulado-issa}{%
\subsection{Indicador de Suficiencia Sanitaria acumulado
(ISSa)}\label{indicador-de-suficiencia-sanitaria-acumulado-issa}}

El ISSa está definido en términos de porcentaje. Desde esa perspectiva,
valores altos del ISSa (cercanos a 100) indicarán que el sistema
sanitario está en disposición de absorber el flujo de diagnósticos
confirmados de coronavirus para que puedan ser tratados según los
protocolos establecidos. Por el contrario, valores bajos del ISSa
(cercanos a 0) sintomatizarán limitaciones que impedirán la correcta
gestión del flujo de casos diagnosticados. Tomando como referencia un
análisis de los valores del ISSa de la primera ola, se deben alcanzar
valores de este indicador por encima del 95\% para poder dar por
controlada la situación de pandemia desde un punto de vista de
suficiencia sanitaria.

Del ISSa se presentan dos estimaciones:

\begin{itemize}
\item
  Una primera estimación de su valor desde el inicio de la segunda ola
  -datada a principios de julio- hasta la actualidad. \textbf{En estos
  momentos, el ISSa de España correspondiente a la segunda fase de la
  oleada y toma un valor del 83.40\%} (ver tabla 1), con una tendencia
  creciente ya que hace 14 días su valor era del 75.13\% (ver figura 1,
  izquierda). \textbf{Este comportamiento indica que la segunda ola se
  está controlando moderadamente.}
\item
  Una segunda estimación de su valor a lo largo de toda la pandemia, es
  decir, tomando para su cálculo el periodo completo que abarca desde el
  mes de marzo hasta la actualidad (ver figura 1, derecha). Su valor es
  del 84.11\%. Esta estimación permite ver el avance completo de la
  pandemia, pero no refleja la realidad de la segunda ola, ya que los
  datos de la primera ola incrementan el valor del indicador.
\end{itemize}

También se calcula el ISSa para cada comunidad autónoma. Se puede
comprobar en la figura 2 que el ISSa de la segunda oleada es creciente
en la mayor parte de las autonomías. Los valores más elevados del ISSa
se alcanzan en Islas Baleares y Canarias (superiores al 95\%). Comunidad
Valenciana, Cataluña, Galicia y País Vasco están cerca de alcanzar ese
valor. Por tanto, todas ellas están cerca de alcanzar valores de control
de la segunda ola. Los valores más bajos del ISSa en la segunda ola se
dan en Asturias (ISSa = 66.13) y Extremadura (ISSa = 69.4).

\begin{table}[!h]

\caption{\label{tab:tabla}ISS acumulado por CCAA, segunda ola y pandemia.}
\centering
\fontsize{9}{11}\selectfont
\begin{tabular}[t]{l|l|l}
\hline
Comunidad Autónoma & ISSa (\%) 2ª ola & ISSa (\%) pandemia\\
\hline
Andalucia & 82.57 & 87.58\\
\hline
Aragón & 85.37 & 88.21\\
\hline
Asturias & 66.13 & 83.69\\
\hline
Baleares & > 95 & > 95\\
\hline
C. Valenciana & 93.27 & > 95\\
\hline
Canarias & > 95 & > 95\\
\hline
Cantabria & 89.62 & 94.25\\
\hline
Castilla La Mancha & 73.45 & 86.85\\
\hline
Castilla y León & 74.34 & 86.83\\
\hline
Cataluña & 94.75 & > 95\\
\hline
Ceuta & 86.12 & 92.19\\
\hline
Extremadura & 69.4 & 82.28\\
\hline
Galicia & 94.79 & > 95\\
\hline
La Rioja & 88.4 & 94.88\\
\hline
Madrid & 76.59 & 84.25\\
\hline
Melilla & 76.15 & 80.7\\
\hline
Murcia & 77.02 & 80.96\\
\hline
Navarra & 72.29 & 82.48\\
\hline
País Vasco & 94.55 & > 95\\
\hline
España & 83.41 & 84.11\\
\hline
\end{tabular}
\end{table}

\begin{figure}
\centering
\includegraphics{20201101_COVID19_DSLAB_informe_ejecutivo_files/figure-latex/fig_iss3-1.pdf}
\caption{\label{fig:fig_iss3} ISS acumulado, segunda ola y pandemia.}
\end{figure}

\vspace{0.2cm}

\begin{figure}
\centering
\includegraphics{20201101_COVID19_DSLAB_informe_ejecutivo_files/figure-latex/fig17a_res-1.pdf}
\caption{\label{fig:fig17a_res} ISS acumulado en las Comunidades
Autónomas, segunda ola.}
\end{figure}

\vspace{0.2cm}

\begin{figure}
\centering
\includegraphics{20201101_COVID19_DSLAB_informe_ejecutivo_files/figure-latex/fig17a_res2-1.pdf}
\caption{\label{fig:fig17a_res} ISS acumulado en las Comunidades
Autónomas, desde el inicio de la pandemia.}
\end{figure}

\newpage

\hypertarget{indicador-de-suficiencia-sanitaria-semanal-isss}{%
\subsection{Indicador de Suficiencia Sanitaria semanal
(ISSs)}\label{indicador-de-suficiencia-sanitaria-semanal-isss}}

El ISSs es un indicador que toma valores positivos, siendo clave superar
un valor de 1 para garantizar la suficiencia del sistema. Sucesivos
valores semanales por debajo de 1 implican posibles saturaciones del
sistema sanitario, mientras que si el indicador se sitúa por encima de 1
evidencia mejores niveles de suficiencia. Para que el sistema sanitario
esté en condiciones de estabilizarse es necesario que el ISSa mantenga
una tendencia creciente y que el ISSs se sitúe de manera sostenida por
encima de 1. \textbf{Desde una perspectiva nacional, el ISSs de la
segunda ola ha superado esta semana el valor de 1 (ISSs = 1.01, ver
tabla 2), también con una sostenida tendencia creciente a lo largo del
mes de septiembre} (ver figura 4). Desagregando por comunidades
autónomas, todas ellas presentan un comportamiento similar (ver figura
5). En el momento este indicador se estabilice por encima de 1, el
número de casos resueltos semanalmente superará al de casos
diagnosticados. \textbf{El actual valor del ISSs en torno a 1 indica que
probablemente nos encontramos en un punto de inflexión en el devenir de
la segunda ola, comenzando a partir de ahora una mejora previsible de la
situación de suficiencia sanitaria tanto a nivel nacional como
autonómico.}

\begin{table}[!h]

\caption{\label{tab:tabla2}ISS semanal por CCAA, segunda ola y pandemia.}
\centering
\fontsize{9}{11}\selectfont
\begin{tabular}[t]{l|r|r}
\hline
Comunidad Autónoma & ISSs 2ª ola & ISSs pandemia\\
\hline
Andalucia & 0.91 & 0.90\\
\hline
Aragón & 0.90 & 0.90\\
\hline
Asturias & 0.51 & 0.46\\
\hline
Baleares & 0.01 & 0.01\\
\hline
C. Valenciana & 1.62 & 1.61\\
\hline
Canarias & 2.29 & 2.29\\
\hline
Cantabria & 2.17 & 2.17\\
\hline
Castilla La Mancha & 0.94 & 0.88\\
\hline
Castilla y León & 0.71 & 0.63\\
\hline
Cataluña & 1.89 & 1.68\\
\hline
Ceuta & 0.99 & 0.99\\
\hline
Extremadura & 0.80 & 0.80\\
\hline
Galicia & 1.06 & 1.05\\
\hline
La Rioja & 0.95 & 0.95\\
\hline
Madrid & 0.78 & 0.78\\
\hline
Melilla & 1.00 & 1.00\\
\hline
Murcia & 1.13 & 1.13\\
\hline
Navarra & 0.93 & 0.93\\
\hline
País Vasco & 1.40 & 1.40\\
\hline
España & 1.01 & 0.98\\
\hline
\end{tabular}
\end{table}

\begin{figure}
\centering
\includegraphics{20201101_COVID19_DSLAB_informe_ejecutivo_files/figure-latex/fig_iss2-1.pdf}
\caption{\label{fig:fig_iss2} ISS semanal, segunda ola y pandemia.}
\end{figure}

\begin{figure}
\centering
\includegraphics{20201101_COVID19_DSLAB_informe_ejecutivo_files/figure-latex/figura_ISSSD_ccaa-1.pdf}
\caption{\label{fig:figura_ISSSD_ccaa} ISS semanal por Comunidades
Autónomas.}
\end{figure}

\clearpage

\hypertarget{estimaciuxf3n-de-casos-activos}{%
\subsection{Estimación de casos
activos}\label{estimaciuxf3n-de-casos-activos}}

Las \textbf{figuras 6 y 7} muestran gráficos que resumen de forma muy
ajustada la evolución de la pandemia en España. En concreto, presentan
\textbf{la estimación del número de casos diagnosticados que permanecen
activos cada semana}, tanto a nivel nacional como por comunidades
autónomas. \textbf{España presenta ya una clara ralentización en el
crecimiento del número de casos activos de la segunda ola. A excepción
de Extremadura, Madrid y Navarra, el resto de comunidades autónomas
muestran escenarios de decrecimiento o de una ralentización del
crecimiento de los casos activos y, en estas tres comunidades autónomas,
el escenario de ralentización es previsible que se alcance en breve.
Recordamos que este escenario se alcanza cuando el ISSs se sitúa en
torno al valor de 1.}

\vspace{0.2cm}

\begin{figure}
\centering
\includegraphics{20201101_COVID19_DSLAB_informe_ejecutivo_files/figure-latex/fig1-1.pdf}
\caption{\label{fig:fig17a_res} Estimación de casos activos en España
desde el inicio de la pandemia.}
\end{figure}

\begin{figure}
\centering
\includegraphics{20201101_COVID19_DSLAB_informe_ejecutivo_files/figure-latex/figura_activos_ccaa-1.pdf}
\caption{\label{fig:figura_activos_ccaa} Estimación de casos activos por
Comunidades Autónomas.}
\end{figure}

\newpage

\hypertarget{definiciones}{%
\subsection{Definiciones}\label{definiciones}}

Definiciones básicas:

\begin{itemize}
\item
  \(Confirmados_s\): número de diagnósticos confirmados acumulados a
  fecha \(s\).
\item
  \(NuevosCasos_{s} = Confirmados_{s} - Confirmados_{s-1}\): nuevos
  contagiados la semana \(s\).
\item
  \(Fallecidos_s\): número de fallecimientos acumulados a fecha \(s\).
\item
  \(FallecidosSemanales_s = Fallecidos_{s} - Fallecidos_{s-1}\): número
  de fallecimientos la semana \(s\).
\item
  \(Recuperados_s\): número de casos recuperados acumulados a fecha
  \(s\).
\item
  \(RecuperadosSemanales_{s} = Recuperados_{s} - Recuperados_{s-1}\):
  número de recuperados en la semana \(s\).
\item
  \(ActivosSemanales_s = Confirmados_s - Fallecidos_s - Recuperados_s\).
  Casos activos al final de la semana \(s\).
\item
  \(AtendidosSemanales_{s} = RecuperadosSemanales_s + FallecidosSemanales_s\):
  Pacientes atendidos la semana \(s\).
\item
  \(Atendidos_{s} = Recuperados_s + Fallecidos_s\): Pacientes atendidos
  acumulados a fecha \(s\).
\item
  \textbf{ISS acumulado a semana \(s\)}
  \[ISS\_Acumulado_s = \frac{Atendidos_s}{Confirmados_s} \times 100\]
  Este indicador mide el estrés acumulado del sistema. Cuanto más
  próximo a 100, más desestresado está el sistema.
\item
  \textbf{ISS Semanal, semana \(s\)}
\end{itemize}

\[
ISS_s = \frac{AtendidosSemanales_s}{NuevosCasos_{s}}
\]

\vspace{4cm}

\begin{center}
\includegraphics{cc.png}
\end{center}

\vspace{0.2cm}

\begin{center}
Esta obra está bajo una licencia de Creative Commons 
\end{center}

\begin{center}
Reconocimiento-CompartirIgual 4.0 Internacional.
\end{center}

\end{document}
