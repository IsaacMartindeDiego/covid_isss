% Options for packages loaded elsewhere
\PassOptionsToPackage{unicode}{hyperref}
\PassOptionsToPackage{hyphens}{url}
%
\documentclass[
  11pt,
]{article}
\usepackage{lmodern}
\usepackage{amssymb,amsmath}
\usepackage{ifxetex,ifluatex}
\ifnum 0\ifxetex 1\fi\ifluatex 1\fi=0 % if pdftex
  \usepackage[T1]{fontenc}
  \usepackage[utf8]{inputenc}
  \usepackage{textcomp} % provide euro and other symbols
\else % if luatex or xetex
  \usepackage{unicode-math}
  \defaultfontfeatures{Scale=MatchLowercase}
  \defaultfontfeatures[\rmfamily]{Ligatures=TeX,Scale=1}
  \setmainfont[]{Roboto Light}
\fi
% Use upquote if available, for straight quotes in verbatim environments
\IfFileExists{upquote.sty}{\usepackage{upquote}}{}
\IfFileExists{microtype.sty}{% use microtype if available
  \usepackage[]{microtype}
  \UseMicrotypeSet[protrusion]{basicmath} % disable protrusion for tt fonts
}{}
\makeatletter
\@ifundefined{KOMAClassName}{% if non-KOMA class
  \IfFileExists{parskip.sty}{%
    \usepackage{parskip}
  }{% else
    \setlength{\parindent}{0pt}
    \setlength{\parskip}{6pt plus 2pt minus 1pt}}
}{% if KOMA class
  \KOMAoptions{parskip=half}}
\makeatother
\usepackage{xcolor}
\IfFileExists{xurl.sty}{\usepackage{xurl}}{} % add URL line breaks if available
\IfFileExists{bookmark.sty}{\usepackage{bookmark}}{\usepackage{hyperref}}
\hypersetup{
  pdftitle={Informe del 04 de marzo de 2021},
  hidelinks,
  pdfcreator={LaTeX via pandoc}}
\urlstyle{same} % disable monospaced font for URLs
\usepackage[margin=1in]{geometry}
\usepackage{graphicx,grffile}
\makeatletter
\def\maxwidth{\ifdim\Gin@nat@width>\linewidth\linewidth\else\Gin@nat@width\fi}
\def\maxheight{\ifdim\Gin@nat@height>\textheight\textheight\else\Gin@nat@height\fi}
\makeatother
% Scale images if necessary, so that they will not overflow the page
% margins by default, and it is still possible to overwrite the defaults
% using explicit options in \includegraphics[width, height, ...]{}
\setkeys{Gin}{width=\maxwidth,height=\maxheight,keepaspectratio}
% Set default figure placement to htbp
\makeatletter
\def\fps@figure{htbp}
\makeatother
\setlength{\emergencystretch}{3em} % prevent overfull lines
\providecommand{\tightlist}{%
  \setlength{\itemsep}{0pt}\setlength{\parskip}{0pt}}
\setcounter{secnumdepth}{-\maxdimen} % remove section numbering
\usepackage{titling}
\usepackage{fancyhdr}
\usepackage{lastpage}



\topmargin=-0.45in
\evensidemargin=-1in
\oddsidemargin=0in
\textwidth=6.5in
\textheight=9.0in
\headsep=0.25in

\linespread{1}
 \pagestyle{fancy}
 \fancyhead[LO,LE]{\thetitle}
 \fancyhead[RO,RE]{\includegraphics[width=0.5cm]{./DSLab_logo_2.png}}
 \fancyfoot[LO,LE]{DSLab, methaodos.org, Academia Joven de España \thedate}
 \cfoot{}
 \fancyfoot[RE,RO]{Página\ \thepage\ de\ \protect\pageref*{LastPage}}

\renewcommand\headrulewidth{0.4pt}
\renewcommand\footrulewidth{0.4pt}


\pretitle{%
  \begin{center}
  \LARGE
  \vspace{-2.5cm}
  \includegraphics[width=3.2cm]{./methaodos.png}
  \hspace{3cm}
  \includegraphics[width=3.5cm]{./dslab.png}
  \hspace{3cm}
  \includegraphics[width=3cm]{./aje.png}\\ 
\vspace{1cm}
\textbf{Suficiencia Sanitaria y COVID-19}\\
}
\posttitle{\end{center}}
\usepackage[font=small,skip=8pt]{caption}
\usepackage{float}
\floatplacement{figure}{H}
\renewcommand{\contentsname}{Índice}
\usepackage{booktabs}
\usepackage{longtable}
\usepackage{array}
\usepackage{multirow}
\usepackage{wrapfig}
\usepackage{float}
\usepackage{colortbl}
\usepackage{pdflscape}
\usepackage{tabu}
\usepackage{threeparttable}
\usepackage{threeparttablex}
\usepackage[normalem]{ulem}
\usepackage{makecell}
\usepackage{xcolor}

\title{Informe del 04 de marzo de 2021}
\author{}
\date{\vspace{-2.5em}}

\begin{document}
\maketitle

\renewcommand{\figurename}{Figura}
\renewcommand{\tablename}{Tabla}

\tableofcontents

\vspace{.5cm}

\begin{center}\rule{0.5\linewidth}{0.5pt}\end{center}

\vspace{.5cm}

Para analizar el comportamiento de la segunda ola de la pandemia
provocada por el virus SARS-CoV-2 y la enfermedad COVID-19 en España, se
han diseñado dos \textbf{Indicadores de Suficiencia Sanitaria (ISS):}
uno acumulado \textbf{(ISSa)} y otro semanal \textbf{(ISSs)}. Además, se
lleva a cabo una estimación del número de \textbf{casos activos} desde
que se inició la pandemia.

\hypertarget{indicador-de-suficiencia-sanitaria-acumulado-issa}{%
\subsection{Indicador de Suficiencia Sanitaria acumulado
(ISSa)}\label{indicador-de-suficiencia-sanitaria-acumulado-issa}}

El ISSa está definido en términos de porcentaje. Desde esa perspectiva,
valores altos del ISSa (cercanos a 100) indicarán que el sistema
sanitario está en disposición de absorber el flujo de diagnósticos
confirmados de coronavirus para que puedan ser tratados según los
protocolos establecidos. Por el contrario, valores bajos del ISSa
(cercanos a 0) sintomatizarán limitaciones que impedirán la correcta
gestión del flujo de casos diagnosticados. Tomando como referencia un
análisis de los valores del ISSa de la primera ola, se deben alcanzar
valores de este indicador por encima del 95\% para poder dar por
controlada la situación de pandemia desde un punto de vista de
suficiencia sanitaria.

Del ISSa se presentan dos estimaciones:

\begin{itemize}
\item
  Una primera estimación de su valor desde el inicio de la segunda ola
  -datada a principios de julio- hasta la actualidad. \textbf{En estos
  momentos, el ISSa de España correspondiente a la segunda ola de la
  pandemia ha vuelto a superar el valor del 95\%} (ver tabla 1), con una
  tendencia claramente creciente (ver figura 1, izquierda). Se confirma,
  por tanto, el cambio de tendencia anunciado hace cuatro semanas.
  \textbf{Este comportamiento del valor del indicador es coherente con
  la consistencia temporal de las medidas, mayoritariamente reactivas,
  que se están tomando a lo largo de la segunda ola, y especialmente las
  tomadas después del periodo navideño.} Entendemos por \textbf{medidas
  reactivas} aquellas que se toman para dar respuesta a valores extremos
  de determinados indicadores (como por ejemplo la tasa de incidencia a
  14 días). El denominador común de estas medidas es que se relajan una
  vez los valores de los indicadores dejan de ser extremos.
\item
  Una segunda estimación de su valor a lo largo de toda la pandemia, es
  decir, tomando para su cálculo el periodo completo que abarca desde el
  mes de marzo hasta la actualidad (ver figura 1, derecha). Su valor
  supera también el 95\%. Esta estimación permite ver el avance completo
  de la pandemia.
\end{itemize}

También se calcula el ISSa para cada comunidad autónoma. Se puede
comprobar en la figura 2 que el ISSa de la segunda oleada tiene también
un comportamiento creciente en todas las autonomías desde hace cuatro
semanas. De hecho, en todas las comunidades autónomas el indicador toma
valores por encima del 95\%.

\begin{table}[!h]

\caption{\label{tab:tabla}ISS acumulado por CCAA, segunda ola y pandemia.}
\centering
\fontsize{9}{11}\selectfont
\begin{tabular}[t]{l|l|l}
\hline
Comunidad Autónoma & ISSa (\%) 2ª ola & ISSa (\%) pandemia\\
\hline
Andalucia & > 95 & > 95\\
\hline
Aragón & > 95 & > 95\\
\hline
Asturias & > 95 & > 95\\
\hline
Baleares & > 95 & > 95\\
\hline
C. Valenciana & > 95 & > 95\\
\hline
Canarias & > 95 & > 95\\
\hline
Cantabria & > 95 & > 95\\
\hline
Castilla La Mancha & > 95 & > 95\\
\hline
Castilla y León & > 95 & > 95\\
\hline
Cataluña & > 95 & > 95\\
\hline
Ceuta & > 95 & > 95\\
\hline
Extremadura & > 95 & > 95\\
\hline
Galicia & > 95 & > 95\\
\hline
La Rioja & > 95 & > 95\\
\hline
Madrid & > 95 & > 95\\
\hline
Melilla & > 95 & > 95\\
\hline
Murcia & > 95 & > 95\\
\hline
Navarra & > 95 & > 95\\
\hline
País Vasco & > 95 & > 95\\
\hline
España & > 95 & > 95\\
\hline
\end{tabular}
\end{table}

\begin{figure}
\centering
\includegraphics{20210304_COVID19_DSLAB_informe_ejecutivo_files/figure-latex/fig_iss3-1.pdf}
\caption{\label{fig:fig_iss3} ISS acumulado, segunda ola y pandemia.}
\end{figure}

\vspace{0.2cm}

\begin{figure}
\centering
\includegraphics{20210304_COVID19_DSLAB_informe_ejecutivo_files/figure-latex/fig17a_res-1.pdf}
\caption{\label{fig:fig17a_res} ISS acumulado en las Comunidades
Autónomas, segunda ola.}
\end{figure}

\vspace{0.2cm}

\begin{figure}
\centering
\includegraphics{20210304_COVID19_DSLAB_informe_ejecutivo_files/figure-latex/fig17a_res2-1.pdf}
\caption{\label{fig:fig17a_res} ISS acumulado en las Comunidades
Autónomas, desde el inicio de la pandemia.}
\end{figure}

\newpage
\setcounter{page}{6}

\hypertarget{indicador-de-suficiencia-sanitaria-semanal-isss}{%
\subsection{Indicador de Suficiencia Sanitaria semanal
(ISSs)}\label{indicador-de-suficiencia-sanitaria-semanal-isss}}

El ISSs es un indicador que toma valores positivos, siendo clave superar
un valor de 1 para garantizar la suficiencia del sistema. Sucesivos
valores semanales por debajo de 1 implican posibles saturaciones del
sistema sanitario, mientras que si el indicador se sitúa por encima de 1
evidencia mejores niveles de suficiencia. Para que el sistema sanitario
esté en condiciones de estabilizarse es necesario que el ISSa mantenga
una tendencia creciente y que el ISSs se sitúe de manera sostenida por
encima de 1. \textbf{Desde una perspectiva nacional, el ISSs de la
segunda ola ha vuelto a bajar del valor de 1 (ISSs = 0.28, ver tabla 2),
continuando con la tendencia oscilante en torno a 1 desde el mes de
septiembre (ver figura 4).} Desagregando por comunidades autónomas,
todas ellas presentan un comportamiento similar (ver figura 5). Sin
embargo, durante el último mes el ISSs ha alcanzado valores superiores a
3 durante dos semanas consecutivas, por lo que es normal que su valor se
modere circunstancialmente. \textbf{Esta situación es posiblemente el
anuncio de una moderación en el crecimiento del indicador acumulado
ISSa.}

\begin{table}[!h]

\caption{\label{tab:tabla2}ISS semanal por CCAA, segunda ola y pandemia.}
\centering
\fontsize{9}{11}\selectfont
\begin{tabular}[t]{l|r|r}
\hline
Comunidad Autónoma & ISSs 2ª ola & ISSs pandemia\\
\hline
Andalucia & 0.02 & 0.02\\
\hline
Aragón & 0.04 & 0.04\\
\hline
Asturias & 2.58 & 2.58\\
\hline
Baleares & 0.01 & 0.01\\
\hline
C. Valenciana & 0.02 & 0.02\\
\hline
Canarias & 1.02 & 1.02\\
\hline
Cantabria & 0.30 & 0.30\\
\hline
Castilla La Mancha & 0.03 & 0.03\\
\hline
Castilla y León & 0.03 & 0.03\\
\hline
Cataluña & 0.00 & 0.00\\
\hline
Ceuta & 1.05 & 1.05\\
\hline
Extremadura & 0.05 & 0.05\\
\hline
Galicia & 0.70 & 0.70\\
\hline
La Rioja & 0.03 & 0.03\\
\hline
Madrid & 0.31 & 0.31\\
\hline
Melilla & 0.02 & 0.02\\
\hline
Murcia & 0.04 & 0.04\\
\hline
Navarra & 0.01 & 0.01\\
\hline
País Vasco & 1.44 & 1.44\\
\hline
España & 0.28 & 0.28\\
\hline
\end{tabular}
\end{table}

\begin{figure}
\centering
\includegraphics{20210304_COVID19_DSLAB_informe_ejecutivo_files/figure-latex/fig_iss2-1.pdf}
\caption{\label{fig:fig_iss2} ISS semanal, segunda ola y pandemia.}
\end{figure}

\begin{figure}
\centering
\includegraphics{20210304_COVID19_DSLAB_informe_ejecutivo_files/figure-latex/figura_ISSSD_ccaa-1.pdf}
\caption{\label{fig:figura_ISSSD_ccaa} ISS semanal por Comunidades
Autónomas.}
\end{figure}

\clearpage

\setcounter{page}{9}

\hypertarget{estimaciuxf3n-de-casos-activos}{%
\subsection{Estimación de casos
activos}\label{estimaciuxf3n-de-casos-activos}}

Las \textbf{figuras 6 y 7} muestran tendencias que resumen de forma muy
ajustada la evolución de la pandemia en España. En concreto, presentan
\textbf{la estimación del número de casos diagnosticados que permanecen
activos cada semana}, tanto a nivel nacional como por comunidades
autónomas. \textbf{Todas las comunidades autónomas muestran escenarios
decrecientes en el último mes. El comportamiento creciente de la última
semana se debe a correcciones de la serie histórica de datos. Esta
situación de decrecimiento del número de casos activos es coherente con
el mantenimiento en el tiempo de las medidas reactivas y con la
implantación de medidas preventivas que van surtiendo efecto (como por
ejemplo la vacunación).}

\vspace{0.2cm}

\begin{figure}
\centering
\includegraphics{20210304_COVID19_DSLAB_informe_ejecutivo_files/figure-latex/fig1-1.pdf}
\caption{\label{fig:fig17a_res} Estimación de casos activos en España
desde el inicio de la pandemia.}
\end{figure}

\begin{figure}
\centering
\includegraphics{20210304_COVID19_DSLAB_informe_ejecutivo_files/figure-latex/figura_activos_ccaa-1.pdf}
\caption{\label{fig:figura_activos_ccaa} Estimación de casos activos por
Comunidades Autónomas.}
\end{figure}

\newpage
\setcounter{page}{11}

\hypertarget{definiciones}{%
\subsection{Definiciones}\label{definiciones}}

Definiciones básicas:

\begin{itemize}
\item
  \(Confirmados_s\): número de diagnósticos confirmados acumulados a
  fecha \(s\).
\item
  \(NuevosCasos_{s} = Confirmados_{s} - Confirmados_{s-1}\): nuevos
  contagiados la semana \(s\).
\item
  \(Fallecidos_s\): número de fallecimientos acumulados a fecha \(s\).
\item
  \(FallecidosSemanales_s = Fallecidos_{s} - Fallecidos_{s-1}\): número
  de fallecimientos la semana \(s\).
\item
  \(Recuperados_s\): número de casos recuperados acumulados a fecha
  \(s\).
\item
  \(RecuperadosSemanales_{s} = Recuperados_{s} - Recuperados_{s-1}\):
  número de recuperados en la semana \(s\).
\item
  \(ActivosSemanales_s = Confirmados_s - Fallecidos_s - Recuperados_s\).
  Casos activos al final de la semana \(s\).
\item
  \(AtendidosSemanales_{s} = RecuperadosSemanales_s + FallecidosSemanales_s\):
  Pacientes atendidos la semana \(s\).
\item
  \(Atendidos_{s} = Recuperados_s + Fallecidos_s\): Pacientes atendidos
  acumulados a fecha \(s\).
\item
  \textbf{ISS acumulado a semana \(s\)}
  \[ISS\_Acumulado_s = \frac{Atendidos_s}{Confirmados_s} \times 100\]
  Este indicador mide el estrés acumulado del sistema. Cuanto más
  próximo a 100, más desestresado está el sistema.
\item
  \textbf{ISS Semanal, semana \(s\)}
\end{itemize}

\[
ISS_s = \frac{AtendidosSemanales_s}{NuevosCasos_{s}}
\]

\vspace{4cm}

\begin{center}
\includegraphics{cc.png}
\end{center}

\vspace{0.2cm}

\begin{center}
Esta obra está bajo una licencia de Creative Commons 
\end{center}

\begin{center}
Reconocimiento-CompartirIgual 4.0 Internacional.
\end{center}

\end{document}
